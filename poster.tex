\documentclass[11pt]{article}
\usepackage{fullpage}
\usepackage[compact]{titlesec}
\setlength{\parindent}{0pt}
\setlength{\parskip}{.1in}

\newenvironment{denseItemize}{%
\begin{list}{$\bullet$}{\setlength{\itemsep}{0in}\setlength{\parsep}{.05in}}}{\end{list}}

\begin{document}

\section*{Proposers}

\begin{quote}
\begin{minipage}{2.75in}
{\large\bf Jaime Spacco}\\
Department of Computer Science\\
Knox College\\
2 East South St, Galesburg, IL 61401\\
{\tt jspacco@knox.edu}
\end{minipage}
\begin{minipage}{2.75in}
{\large\bf David Hovemeyer}\\
Department of Physical Sciences\\
York College of Pennsylvania\\
441 Country Club Rd, York, PA 17403\\
{\tt dhovemey@ycp.edu}
\end{minipage}
\end{quote}

\section*{Statement of Topic}

{\large CloudCoder: Building a Community for Using and Sharing Programming Exercises}

\section*{Significance and Relevance of the Topic}

Learning to program is difficult.  While some students master
fundamental programming concepts with relatively little difficulty,
most students in introductory programming courses struggle
to master syntax and basic programming techniques.

Practice helps.  [TODO: Evidence that more practice is correlated with
success.]  [TODO: Evidence that time spent with interactive execises
improves learning - problets? intelligent tutoring systems?]

Many interactive programming exercise systems have been developed.
For example, CodeLab, CodingBat, Problets, Practice-It!, just to
name a few.  While all of these systems have been useful, all of
them are to some extent closed ecosystems, for a variety of reasons:

\begin{denseItemize}
\item In commercial services such as CodeLab,
      students must pay to access programming exercises
\item All of the existing systems we know about are closed-source
\item For all of the existing systems we know about,
      self-hosting is not an option
\item Some free services such as Practice-It! may include
      exercises that are not freely redistributable
\end{denseItemize}

Our thesis is that an open ecosystem for programming exercises
is needed.  The features of such an ecosystem would be:

\begin{denseItemize}
\item Open source software, with contributions from the community
      actively solicited
\item A central repository for the sharing of freely-redistibutable exercises
      written by community members
\item Freedom for users of the system to self-host or use a hosting service
\end{denseItemize}

All of these features are motivated by the desire to remove 
barriers to the adoption and use of a programming exercises,
and to make a large repository of high-quality exercises available
for anyone to use.

\section*{Content}

Describe CloudCoder and the exercise repository.

Present data on use of the exercise repository: how many exercises? How often have exercises
been reused between institutions?  What have we learned about exercise sharing and how
instructors can make effective use of available exercises?

\section*{Abstract}

Yeah

\end{document}
