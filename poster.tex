\documentclass[11pt]{article}
\usepackage{fullpage}
\usepackage[compact]{titlesec}
\usepackage{url}

\setlength{\parindent}{0pt}
\setlength{\parskip}{.1in}

\newenvironment{denseItemize}{%
\begin{list}{$\bullet$}{\setlength{\itemsep}{0in}\setlength{\parsep}{.05in}}}{\end{list}}

\begin{document}

\section*{Proposers}

\begin{quote}
\begin{minipage}{3in}
{\large\bf Jaime Spacco}\\
Department of Computer Science\\
Knox College\\
2 E. South St, Galesburg, IL 61401, USA\\
{\tt jspacco@knox.edu}
\end{minipage}
\begin{minipage}{3in}
{\large\bf David Hovemeyer}\\
Department of Physical Sciences\\
York College of Pennsylvania\\
441 Country Club Rd, York, PA 17403, USA\\
{\tt dhovemey@ycp.edu}
\end{minipage}\\
\vskip .1in
\begin{minipage}{3in}
{\large\bf Matthew Hertz}\\
Department of Computer Science\\
Canisius College\\
Buffalo, NY 14208, USA\\
{\tt hertzm@canisius.edu}
\end{minipage}
\begin{minipage}{3in}
{\large\bf Paul Denny}\\
Department of Computer Science\\
University of Auckland\\
Private Bag 92019, Auckland, NZ\\
{\tt paul@cs.auckland.ac.nz}
\end{minipage}\\

\end{quote}

\section*{Statement of Topic}

{\large CloudCoder: Building a Community for Using and Sharing Programming Exercises}

\section*{Significance and Relevance of the Topic}

Learning to program is difficult.  While some students master
fundamental programming concepts with relatively little difficulty,
most students in introductory programming courses struggle
to master syntax and basic programming techniques.

Programming exercises---short, focused, automatically assessed
programming problems---can help students learn basic skills and techniques.
Because they are automatically assessed, they provide immediate feedback
for students.  They can be used as self-tests (to accompany reading assignments),
in-class assessment, and for practice and skills development.

Many interactive programming exercise systems have been developed.
For example, CodeLab, CodingBat, Problets, Practice-It!, just to
name a few.  While all of these systems are useful, all of
them are to some extent closed ecosystems, for a variety of reasons:

\begin{denseItemize}
\item In commercial services such as CodeLab,
      students must pay to access programming exercises
\item All of the existing systems we know about are closed-source
\item For all of the existing systems we know about,
      self-hosting is not an option
\item Some free services such as Practice-It! may include
      exercises that are not freely redistributable
\end{denseItemize}

Our thesis is that an open ecosystem for programming exercises
is needed.  The features of such an ecosystem would be:

\begin{denseItemize}
\item Open source software, with contributions from the community
      actively solicited
\item A central repository for the sharing of freely-redistibutable exercises
      written by community members
\item Freedom for users of the system to self-host or use a hosting service
\end{denseItemize}

All of these features are motivated by the desire to remove 
barriers to the adoption and use of a programming exercises,
and to make a large repository of high-quality exercises available
for anyone to use.

\section*{Content}

CloudCoder (\url{http://cloudcoder.org}) is an open source web-based
programming exercise system.  It currently supports the C, Java, and
Python programming languages.  Although in a somewhat early stage of
development, it has been used successfully by several hundred students
at the authors' home institutions.  It is relatively easy to
install and configure, and requires two Linux servers
(one to host the web application and database, and another to build
and test student submissions.)

A novel feature of CloudCoder is a central exercise repository
(\url{https://cloudcoder.org/repo}) where
instructors can easily find exercises to use in their classes,
and share exercises they have written with the community.  All exercises
published in the repository are available under a permissive license
such as Creative Commons.

The poster will discuss the motivations for CloudCoder and briefly summarize
its features.  We will talk about our goal of building a self-sustaining
community around CloudCoder and its exercise repository.

In early 2013 we will be conducting a study where we will use CloudCoder
in introductory programming courses for self-test exercises, in-class
assessment, in-class lab activities, and remediation for students who need
extra practice.  We should be able to present some initial data regarding
two of our primary research questions:

\begin{enumerate}
\item Can CloudCoder help improve student learning?
\item Can exercises shared in the repository be transferred
      easily between institutions and courses?
\end{enumerate}

Finally, we intend to use the poster to solicit additional users.
Our intention is to present a case that there is no reason {\em not}
to use programming exercises in introductory programming courses,
and that CloudCoder provides a simple and free way to add them
to a course.

\section*{Abstract}

Automatically-tested online programming exercises can be useful in
introductory programming courses as self-tests to accompany readings,
for in-class assessment, for skills development, and to provide
additional practice for students who need it.  CloudCoder (\url{http://cloudcoder.org}) is
an effort to build a community based on an open-source programming exercise
system (currently supporting C, Java, and Python) tightly integrated
with a repository of freely-redistributable
programming exercises written and used by members of the community.
The goal of the project is to make programming exercises easy and free
to incorporate into any programming course.

\end{document}
